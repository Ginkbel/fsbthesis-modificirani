\chapter{Uvod}

Ovo poglavlje poslužit će za uvod u problem koji se rješava u ovom
radu, u slučaju doktorske disertacije ovdje se uobičajeno postavlja odgovarajuća
hipoteza.

\section{Primjer podpoglavlja}
Korištena literatura se popisuje u popisu literature pod poglavljem
``Literatura'', a svaka od njih mora biti citirana bar jednom u tekstu, kao
npr.\cite{Siegwart2004}. Literatura mora biti popisana po redoslijedu
pojavljivanja u tekstu za što se brine sami \LaTeX. 

Tekst tekst tekst tekst tekst tekst primjer reference
, i jo\v{s} jedan citat \cite{cern_tim}.
Tekst tekst tekst tekst tekst tekst.

\subsection{Primjer dubljeg strukturiranja teksta}
Slijedi prvi primjer slike (pogl.sliku~\ref{fig1})
\begin{figure}[H]
  \centering
  \includegraphics[height=1.2cm]{01_Uvod/fsb_logo_n}
  \hangcaption{Primjer slike -- logo FSB-a; kod slika primijenjeno je
  zaglavlje s uvlačenjem, \emph{vise\'{c}e} zaglavlje s paketom \emph{hangcaption}}
  \label{fig1}
\end{figure}
\nomenclature[Kf]{$FSB$}{Fakultet strojarstva i brodogradnje}%
%
Slijedi drugi mali primjer slike (pogledaj sliku~\ref{fig2}). U pravilu, na
svaku se koja se pojavljuje treba pozvati u tekstu.
\begin{figure}[H]
  \centering
  \includegraphics[height=3.2cm]{01_Uvod/unizg_plavi_t2}\\
  \hangcaption{Primjer slike -- logo Sveučilišta u Zagrebu.}
  \label{fig2}
\end{figure}

Pored slike dan je i primjer tablice (\ref{tablica}). Uobičajen je stil da se
za tablice zaglavlje piše iznad same tablice (za razliku od slika). 
\begin{table}[H]
\hangcaption{Primjer tablice}
\label{tablica}
  \centering
\begin{tabular}{|c|c|} \hline
    $x$ & $i$ \\\hline\hline
    A & 1 \\
    B & 2 \\
    C & 3 \\
    D & 4 \\ \hline
\end{tabular}
\end{table}
